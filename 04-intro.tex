
{\color{orange} Introduction section}

\begin{itemize}
    \item a brief overview
    \item a challenges section
    \item a section about approach
    \item a road map of the rest of the thesis
\end{itemize}


Automating stance evaluation has been suggested as a valuable first step towards assisting human fact checkers to detect inaccurate claims.\cite{UCLMR}


Recently, an unprecedented amount of false infor-
mation has been flooding the Internet with aims
ranging from affecting individual people’s beliefs
and decisions (Mihaylov et al., 2015a,b; Mihaylov
and Nakov, 2016) to influencing major events such
as political elections (Vosoughi et al., 2018). Con-
sequently, manual fact checking has emerged with
the promise to support accurate and unbiased anal-
ysis of public statements.\cite{memory_network}{\color{orange}(Introduction is good for this part!)}

these notions by Claire Wardle from First Draft, 2 misinforma-
tion is “unintentional mistakes such as inaccurate photo cap-
tions, dates, statistics, translations, or when satire is taken
seriously.”, and disinformation is “fabricated or deliberately
manipulated audio/visual context, and also intentionally cre-
ated conspiracy theories or rumours.”.\cite{stace_survey}